\documentclass[brazilian, fleqn]{article}

\usepackage{babel}
\usepackage[utf8]{inputenc}
\usepackage[T1]{fontenc}
\usepackage{lmodern}

\usepackage{amssymb,amsfonts,amsmath}

\usepackage{tikz}
\usetikzlibrary{calc}

\usepackage[left=2cm, bottom=2cm, right=1.5cm, top=1.5cm]{geometry}

\usepackage{siunitx}
\sisetup{locale = FR}

\usepackage{tcolorbox}
\tcbuselibrary{skins}
\tcbset{boxrule=0pt, top=0pt, bottom=0pt, skin=bicolor, interior style={left color=black!10}}

\DeclareMathOperator{\sen}{sen}
\DeclareMathOperator{\tg}{tg}

\begin{document}
\section*{\centering Questão de aluno 30/04/2023}
\begin{itemize}
    \item Temos que \(T(x,y)=(x+1,y)\)
    \item Para ser linear, \textit{se eu não me engano}, temos que ter \(T(\alpha u + \beta v)=\alpha T(u)+\beta T(v)\)
    \item Sendo assim, vamos testar:
        \begin{gather}
            u=(x_1,y_1) \\
            v=(x_2,y_2) \\
            \alpha u + \beta v = (\alpha x_1 +  \beta x_2, \alpha y_1 + \beta y_2) \\
            T(u)=T(x_1,y_1)=(x_1+1,y_1) \label{eq:1} \\
            T(v)=T(x_2,y_2)=(x_2+1,y_2) \label{eq:2} \\
            T(\alpha u + \beta v) = T(\alpha x_1 +  \beta x_2, \alpha y_1 + \beta y_2) =
            (\alpha x_1 + \beta x_2 + 1, \alpha y_1 + \beta y_2) = 
            \alpha (x_1, y_1) + \beta (x_2,y_2) + (1,0) \label{eq:3}
        \end{gather}
    \item Como pode ser visto, \eqref{eq:3} não é igual a soma de \eqref{eq:1} e \eqref{eq:2} multiplicados por
        \(\alpha\) e \(\beta\), respectivamente
\end{itemize}
\end{document}


\begin{frame}
    \frametitle{Sistemas de equações lineares}
    Um sistema de equações lineares consiste em um conjunto de $m$ equações lineares com $n$ variáveis de $x_i$:

    \begin{align*}
        a_{11} x_1 + a_{12} x_2 + a_{13} x_3 + \cdots  + a_{1n} x_n & =  b_1 \\
        a_{21} x_1 + a_{22} x_2 + a_{23} x_3 + \cdots + a_{2n} x_n & =  b_2 \\
        a_{31} x_1 + a_{32} x_2 + a_{33} x_3 + \cdots + a_{3n} x_n & = b_3 \\
                                                                   & ~\vdots~ \\
        a_{m1} x_1 + a_{m2} x_2 + a_{m3} x_3 + \cdots  + a_{mn} x_n & =  b_m
    \end{align*}
\end{frame}

\begin{frame}
    \frametitle{Sistema de equações lineares - forma matricial}
    O sistema anterior pode ser representado na forma matricial por:

    \[
        \begin{bmatrix}
            a_{11} & a_{12} & a_{13} & \cdots & a_{1n}  \\
            a_{21} & a_{22} & a_{23} & \cdots & a_{2n}  \\
            a_{31} & a_{32} & a_{33} & \cdots & a_{3n}  \\
                   &        &        & \vdots &         \\
            a_{m1} & a_{m2} & a_{m3} & \cdots & a_{mn}
        \end{bmatrix}
        \begin{bmatrix}
            x_1 \\ x_2 \\ x_3 \\ \vdots \\ x_n
        \end{bmatrix}=
        \begin{bmatrix}
            b_1 \\ b_2 \\ b_3 \\ \vdots \\ b_m
        \end{bmatrix}
    \]

    ou simplesmente $Ax=b$, onde $A$ é chamada de matriz de coeficientes, $x$ é o vetor solução e $b$ é o vetor dos termos independentes. Resolver um sistema consiste em encontrar um vetor $x$ que satisfaça simultaneamente às equações.

    \pause 

    Se $A$ for uma matriz quadrada ($m = n$) não singular ($det(A) \neq 0$), então:

    \[
        Ax=b ~\rightarrow~ A^{-1}Ax=A^{-1}b ~\rightarrow~ x=A^{-1}b
    \]
\end{frame}

\begin{frame}
\frametitle{Sistema triangular superior}

Seja um sistema triangular superior 3x3
\[
\begin{bmatrix}
a_{11} & a_{12} & a_{13} \\
0      & a_{22} & a_{23} \\
0      & 0      & a_{33}
\end{bmatrix}
\begin{bmatrix}
x_1 \\ x_2 \\ x_3
\end{bmatrix}
=
\begin{bmatrix}
b_1 \\ b_2 \\ b_3
\end{bmatrix}
\]

O vetor solução \( x\) pode ser obtido através de substituições retroativas:
\begin{align*}
x_3 &= \frac{b_3}{a_{33}} \\
x_2 & = \frac{b_2 - a_{23}x_3}{a_{22}} \\
x_1 &= \frac{b_1 - a_{12}x_2 - a_{13}x_3}{a_{11}}
\end{align*}
\end{frame}

\begin{frame}
Ou seja,

\[
x_i = \frac{b_i - \sum_{j=i+1}^{3} {a_{ij} x_j}}{a_{ii}}
\]
onde \(i = 3, 2, 1\). Para um sistema \(n\) x \(n\), temos
\[
x_i = \frac{b_i - \sum_{j=i+1}^{n} {a_{ij} x_j}}{a_{ii}}
\]
onde \(i = n, n-1, n-2, \ldots , 1\)

\begin{block}{Eliminação de Gauss}
O método de eliminação de Gauss consiste em transformar um sistema de equações lineares em um sistema triangular superior equivalente e obter o vetor solução através de substituições retroativas
\end{block}
\end{frame}

\begin{frame}
Um sistema linear pode ser transformado em outro equivalente através das operações
\begin{itemize}
\item Trocar a ordem de duas equações
\item Multiplicar uma equação por uma constante não nula
\item Somar duas equações
\end{itemize}

\begin{block}
{Exemplo}
Resolva o seguinte sistema usando eliminação de Gauss

\[
\begin{bmatrix}
10 & 3 & -2 \\ 2 & 8 & -1 \\ 1 & 1 & 5
\end{bmatrix}
\begin{bmatrix}
x_1 \\ x_2 \\ x_3
\end{bmatrix}
=
\begin{bmatrix}
57 \\ 20 \\ -4
\end{bmatrix}
\]

\end{block}
\end{frame}

\begin{frame}
\frametitle{Primeira etapa}
Como as operações envolvem as equações, tanto \(A\) quanto \(b\) são modificadas. Assim, é interessante construir uma matriz ''aumentada'' \([A,b]\):

\[
[A,b] =
\left[
\begin{array}{ccc|c}
10 & 3 & -2 & 57 \\ 2 & 8 & -1 & 20 \\ 1 & 1 & 5 & -4
\end{array}
\right]
\]

de maneira que as operações ocorrendo em \(A\) também ocorram em \(b\). 

\begin{block}
{}
Em uma matriz triangular superior, todos os elementos abaixo da diagonal são nulos. Assim, devemos zerar todos os elementos abaixo da diagonal
\end{block}

\end{frame}

\begin{frame}
\frametitle{Segunda etapa}

\begin{enumerate}
\item ''Somamos'' a segunda equação (\(L_2\)) com a primeira (\(L_1\)) multiplicada por uma constante, de forma que o elemento \(a_{21} = 0\)
\[
L_2 \leftarrow L_2 - \frac{a_{21}}{a_{11}} L_1
\] 
\item ''Somamos'' a terceira equação (\(L_3\)) com a primeira (\(L_1\)) multiplicada por uma constante, de forma que o elemento \(a_{31} = 0\)
\[
L_3 \leftarrow L_3 - \frac{a_{31}}{a_{11}} L_1
\] 
\end{enumerate}
\[
\left[
\begin{array}{ccc|c}
10 & 3 & -2 & 57 \\ 2 & 8 & -1 & 20 \\ 1 & 1 & 5 & -4
\end{array}
\right]
\rightarrow
\left[
\begin{array}{ccc|c}
10 & 3 & -2 & 57 \\ 0 & 37/5 & -3/5 & 43/5 \\ 0 & 7/10 & 26/5 & -97/10
\end{array}
\right]
\]
\end{frame}

\begin{frame}
\frametitle{Terceira etapa}
\begin{enumerate}
\item ''Somamos'' a terceira equação (\(L_3\)) com a segunda (\(L_2\)) multiplicada por uma constante, de forma que o elemento \(a_{32} = 0\)
\[
L_3 \leftarrow L_3 - \frac{a_{32}}{a_{22}} L_2
\] 
\end{enumerate}
\[
\left[
\begin{array}{ccc|c}
10 & 3 & -2 & 57 \\ 0 & 37/5 & -3/5 & 43/5 \\ 0 & 7/10 & 26/5 & -97/10
\end{array}
\right]
\rightarrow
\left[
\begin{array}{ccc|c}
10 & 3 & -2 & 57 \\ 0 & 37/5 & -3/5 & 43/5 \\ 0 & 0 & 389/74 & -389/37
\end{array}
\right]
\]
\end{frame}

\begin{frame}
\frametitle{Quarta etapa}
Resolvemos o sistema triangular superior usando a expressão
\[
x_i = \frac{b_i - \sum_{j=i+1}^{n} {a_{ij} x_j}}{a_{ii}}
\]
onde \(i = n, n-1, n-2, \ldots , 1\)

\begin{align*}
x_3 &= \frac{-389/37}{389/74}=-2 \\
x_2 & =\frac{43/5 - (-3/5)*(-2)}{37/5}=1 \\
x_1 &= \frac{57-3*1-(-2)*(-2)}{10} = 5
\end{align*}

\end{frame}

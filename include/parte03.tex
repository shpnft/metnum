\newcommand{\xpartcolor}[1]{\textcolor{blue}{#1}}
\newcommand{\ypartcolor}[1]{\textcolor{red}{#1}}

\newcommand{\funcao}[2]{
    f\left(\xpartcolor{#1},~\ypartcolor{#2}\right)
}
\newcolumntype{a}{>{\columncolor{black!20}}c}

\begin{frame}<1>[label=definicao]{Método de Runge Kutta de 4\textordfeminine{} ordem clássico}
    Seja o problema de valor inicial (PVI):
    \begin{align*}
        y^\prime &= \funcao{x}{y}\quad y(x_0)=y_0 \quad y(x_N)=?
        \onslide<2->{
            \\
            \intertext{temos que:}
        k_1 &= \funcao{x}{y} \\
        k_2 &= \funcao{x+\frac{h}{2}}{y+\frac{k_1h}{2}} \\
        k_3 &= \funcao{x+\frac{h}{2}}{y+\frac{k_2h}{2}} \\
        k_4 &= \funcao{x+h}{y+k_3h} \\
        x_{i+1} &= x_i+h \\
        y_{i+1} &= y_i+\frac{h}{6}(k_1+2k_2+2k_3+k4)
    }
\end{align*}
\pause onde \(i=0,\ldots,N-1\) e \(h=(x_N-x_0)/N\)
\end{frame}

\begin{frame}<2-4>[label=exemploone]{Exemplo 1}
    Seja o problema de valor inicial (PVI):
    \begin{center}
        \tikz [remember picture, baseline] {
            \node (t1) {
                \(
                \displaystyle
                y'=1+\frac{y}{x}+\left(\frac{y}{x}\right)^2 
                \)
            };
        }
        \tikz [remember picture, baseline] {
            \node (t2) {
                \(
                \displaystyle
                y(1)=0 
                \)
            };
        }
        \tikz [remember picture, baseline] {
            \node (t3) {
                \(
                \displaystyle
                y(3)=? 
                \)
            };
        }
    \end{center}

    \begin{tikzpicture}[overlay, remember picture]
        \foreach \x/\s/\t in {
            t1/2/{Equação diferencial ordinária, EDO}, 
            t2/3/Condição inicial, 
        t3/4/Valor procurado}{
            \only<\s> {
                \draw [fill=red, opacity=0.1] (\x.north west) rectangle (\x.south east);
                \draw [->] (\x.south) -- ++ (0,-1) node [below] {\t};
            }
        }
    \end{tikzpicture}

    \only<5-> {\alert{Solução:}}

    \begin{align*}
        \only<5> {k_1 &= \funcao{x}{y} \quad N=100\\}
        \only<6> {k_2 &= \funcao{x+\frac{h}{2}}{y+\frac{k_1h}{2}} \\}
        \only<7> {k_3 &= \funcao{x+\frac{h}{2}}{y+\frac{k_2h}{2}} \\}
        \only<8> {k_4 &= \funcao{x+h}{y+k_3h} \\}
        \only<9> {
        x_{i+1} &= x_i+h \\
        y_{i+1} &= y_i+\frac{h}{6}(k_1+2k_2+2k_3+k4)
    }
\end{align*}

\begin{center}
    \only<5-8> {
        \begin{tabular}{a|p{4cm}|p{4cm}|p{4cm}}
            \rowcolor{black!20} 
            \only<5>{ 
            & A & B & C \\ \hline
            1 & h= & =(\xpartcolor{3}-\xpartcolor{1})/100 & \\ \hline
            2 & x & y & k1 \\ \hline
            3 & \xpartcolor{1} & \ypartcolor{0} & 
            =1+\ypartcolor{B3}/\xpartcolor{A3}+(\ypartcolor{B3}/\xpartcolor{A3})\^{}2 \\ 
        }
        \only<6>{ 
            & D & E & F \\ \hline
    1 &  &  & \\ \hline
    2 & x2 & y2 & k2 \\ \hline
    3 & =\xpartcolor{A3+\$B\$1/2} & 
    =\ypartcolor{B3 + C3*\$B\$1/2} & 
    =1+\ypartcolor{E3}/\xpartcolor{D3}+(\ypartcolor{E3}/\xpartcolor{D3})\^{}2 \\ 
}
\only<7>{ 
            & G & H & I \\ \hline
1 &  &  & \\ \hline
2 & x3 & y3 & k3 \\ \hline
3 & =\xpartcolor{A3+\$B\$1/2} & 
=\ypartcolor{B3 + F3*\$B\$1/2} & 
=1+\ypartcolor{H3}/\xpartcolor{G3}+(\ypartcolor{H3}/\xpartcolor{G3})\^{}2 \\ 
}
\only<8>{ 
            & J & K & L \\ \hline
1 &  &  & \\ \hline
2 & x4 & y4 & k4 \\ \hline
3 & =\xpartcolor{A3+\$B\$1} & 
=\ypartcolor{B3 + I3*\$B\$1} & 
=1+\ypartcolor{K3}/\xpartcolor{J3}+(\ypartcolor{K3}/\xpartcolor{J3})\^{}2 \\ 
}
\end{tabular}
}
\only<9>{
    \begin{tabular}{a|p{4cm}|p{8cm}}
        \rowcolor{black!20}
        & A & B  \\ \hline
        1 & h= & =(\xpartcolor{3}-\xpartcolor{1})/100  \\ \hline
        2 & x & y  \\ \hline
        3 & \xpartcolor{1} & \ypartcolor{0}  \\ \hline
        4 & =\xpartcolor{A3+\$B\$1} &
        =\ypartcolor{B3+\$B\$1/6*(C3+2*F3+2*I3+L3)} \\
    \end{tabular}
}
\end{center}
\end{frame}

\againframe<2->{definicao}

\begin{frame}{}
    \begin{center}
        \begin{tikzpicture}[bob 1/.style={above,yshift=1ex}, bob 2/.style={below,yshift=-1ex}] 
            \foreach \x in {0,...,7} {
                \draw[thick,fill=black] (\x,0) circle (1pt) 
                    node [bob 1] (y) {\(y_{\x}\)}
                    node [bob 2] {\(x_{\x}\)}
                    -- +(1,0);
                \only<+> {\draw [->, thick, red] (y.north) to [out=30, in=150] +(1,0);}
                }
                \draw[thick, fill=black, dotted] (8,0) 
                    circle (1pt) 
                    node [bob 1] (y) {\(y_{8}\)} 
                    node [bob 2] {\(x_{8}\)}
                    -- (10,0) 
                    circle (1pt)
                    node [bob 1] (dez) {\(y_{N}\)} 
                    node [bob 2] {\(x_{N}\)};
                %FIXME: vimtex não consegue indentar isso aqui...
                \only<+> {\draw [->, thick, red] (y.north) to [out=30, in=150] +(1,0);}
                    \only<+> {\draw [->, thick, red] ($(dez.north)-(1,0)$) to [out=30, in=150] (dez.north);}
                        \only<+-> {\node [star, star points=5, fill=red!25, bob 1] at (10,0) {\(y_{N}\)};}
                        \end{tikzpicture}%   
                    \end{center} 

                    \onslide<+->{
                        \begin{block}
                            {Decisões}
                            \begin{itemize}
                                \item Temos que \(h=(x_N-x_0)/N\)
                                \item Quanto maior o valor de \(N\), melhor o método se comporta em um \textit{mundo ideal}
                                \item Mas quanto maior o valor de \(N\) menor é o valor de \(h\) e chega um ponto onde \textbf{não há casas decimais o suficiente armazenadas na memória do computador}
                                \item Por exemplo, \num{3}+\num{0.0000000000001}=\num{3} se o computador só guarda 9 casas decimais, causando \textit{erros de arredondamento}
                            \end{itemize}
                        \end{block}
                    }
                \end{frame}

                \againframe<5->{exemploone}

                \begin{frame}{Finalmente...}

                    \begin{itemize}
                        \item Copia-se C3:L3 em C4:L4
                        \item Copia-se A4:L4 em A5:L103
                        \item O valor procurado está em B103
                        \item \alert{Observação:} se \(N \neq 100\), troca-se 103 por \(N+3\)
                    \end{itemize}

                \end{frame}

                \begin{frame}{Mais exemplos...}
                    Determine o valor procurado em cada um dos seguintes PVI's:

                    \begin{enumerate}\setcounter{enumi}{1}
                        \item \(y'=1+(x-y)^2 \quad y(2)=1 \quad y(3)=?\)
                        \item \(y'=1+y/x \quad y(1)=2 \quad y(2)=?\)
                        \item \(y'=y^2/(1+x) \quad y(1)=-1/\ln{2} \quad y(2)=?\)
                    \end{enumerate}

                \end{frame}
